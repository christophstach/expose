\documentclass[]{article}

%Deutsche Bezeichnungen für angezeigte Namen (z.B. Innhaltsverzeichnis etc.)
\usepackage[ngerman]{babel}

% prints dummy text
\usepackage{blindtext}

% Ränder bei Bedarf zeigen
%\usepackage{showframe}

%Erlaubt unteranderem Umbrücke captions
\usepackage{caption}

%Kompakte Listen
\usepackage{paralist}

%Zitate besser formatieren und darstellen
\usepackage{epigraph}

%Zeilenabstand 1,5
\usepackage[onehalfspacing]{setspace}

%Verbesserte Darstellung der Buchstaben zueinander
\usepackage[stretch=10]{microtype}

%Unterstützung von Umlauten und anderen Sonderzeichen (UTF-8)
\usepackage{lmodern}
\usepackage[utf8]{luainputenc}
\usepackage[T1]{fontenc}

%Einfachere Zitate
\usepackage{epigraph}

%Verwendung von Akronymen
\usepackage[printonlyused]{acronym}

%Unterstützung der H positionierung (keine automatische Verschiebung eingefügter Elemente)
\usepackage{float} 

%Erlaubt Umbrüche innerhalb von Tabellen
\usepackage{tabularx}

%Erlaubt Seitenumbrüche mit Tabellen
\usepackage{longtable}

%Erlaubt die Darstellung von Sourcecode mit Highlighting
\usepackage[outputdir=out]{minted}

%Definierung eigener Farben bei nutzung eines selbst vergebene Namens
\usepackage[table,xcdraw]{xcolor}

%Vektorgrafiken
\usepackage{tikz}

%Grafiken (wie jpg, png, etc.)
\usepackage{graphicx}

%Grafiken von Text umlaufen lassen
\usepackage{wrapfig}

%Ermöglicht Verknüpfungen innerhalb des Dokumentes (e.g. for PDF), Links werden durch "hidelink" nicht explizit hervorgehoben
\usepackage[hidelinks, ngerman]{hyperref}

%Einbindung und Verwaltung von Literaturverzeichnissen
\usepackage{csquotes} %wird von biber benötigt
\usepackage[style=alphabetic, backend=biber, bibencoding=utf8]{biblatex}
\addbibresource{references/references.bib}

% for enumeration
\usepackage{enumitem}

%-------------------------------Zusätzliche Anpassungen und Modifikationen--------------------------------------------%
%Pluszeichen in der Referenz beim zitieren ausblenden
\renewcommand*{\labelalphaothers}{}

\addto{\captionsngerman}{\renewcommand{\abstractname}{Abstract}}

%Exposé Type
\newcommand*{\exposeType}[1]{\gdef\@exposeType{#1}}